\documentclass{article}

\usepackage{amsmath}
\usepackage{geometry}

\newcommand{\norm}[1]{\left\lVert\,#1\,\right\rVert}
\newcommand{\vecb}[1]{\overrightarrow{#1}}

\title{{\Huge \textbf{Assignment 1}}}
\author{Ding Markus \and Mamie Robin \and Montial Charline}

\begin{document}
    \maketitle

    \section{Deriving the implicit formula for the cylinder}
    
    To get the implicit equation, the following one needs to be solved:
    
    \[ \norm{x - c'} - r = 0 \]
    
    where $x$ is the intersection of the ray and the cylinder, $c'$ is the projection of this point onto the axis of the cylinder and $r$ its radius.
    
    To compute $c'$, let us define $\vec{a}$ as its normalized axis, $c$ as its center and $\vec{b}$ as the vector pointing from $c$ to $x$ -- i.e. $\vec{b} = x - c$.
    By computing the dot product between $\vec{b} \cdot \vec{a}$, we get the distance between $c$ and $c'$ since $\vec{a}$ is normal -- it's an orthogonal projection.
    We get $c'$ by using the following:
    
    \[ c + (\vec{b} \cdot \vec{a}) \, \vec{a} \]
    
    The fact that $\vec{a}$ is already normalized is to be kept in mind.
    We finally get:
    
    \[ \norm{x - c - (\vec{b} \cdot \vec{a}) \, \vec{a}} - r = 0 \]
    
    \begin{align*}
        0 &= \norm{x - c - (\vec{b} \cdot \vec{a}) \, \vec{a}} - r\\
        &= \left(x - c - (\vec{b} \cdot \vec{a}) \, \vec{a}\right)^2 - r^2\\
    \end{align*}
    
    Now, let us replace $x$ with the explicit ray equation $o + t \, \vec{d}$, where $o$ is its origin point and $\vec{d}$ its vector, and $\vec{b}$ with its definition.
    
    \begin{align*}
        0 &= \left(o + t \, \vec{d} - c - ((o + t \, \vec{d} - c) \cdot \vec{a}) \, \vec{a}\right)^2 - r^2\\
        &= \left(t \, \vec{d} + o - c - ((o + t \, \vec{d} - c) \cdot \vec{a}) \, \vec{a}\right)^2 - r^2\\
        &= \left(t \, \vec{d} + o - c - (t \, \vec{d} \cdot \vec{a}) \, \vec{a} - ((o - c) \cdot \vec{a}) \, \vec{a}\right)^2 - r^2\\
        &= \left(t \, (\vec{d} - (\vec{d} \cdot \vec{a}) \, \vec{a}) + o - c - ((o - c) \cdot \vec{a}) \, \vec{a}\right)^2 - r^2
    \end{align*}
    
    Now, let's define $x = \vec{d} - (\vec{d} \cdot \vec{a}) \, \vec{a}$ and $y = o - c - ((o - c) \cdot \vec{a}) \, \vec{a}$.
    We thus get the following:
    
    \begin{align*}
        0 &= \left(t \, x + y\right)^2 - r^2\\
        &= t^2 \, x^2 + t \, 2xy + y^2 - r^2
    \end{align*}
    
    Let's remember that a vector squared is the result of applying the scalar product on itself.
    We finally get as values for our quadratic function:
    
    \begin{align*}
        A = x^2 &= \left(\vec{d} - (\vec{d} \cdot \vec{a}) \, \vec{a}\right) \cdot \left(\vec{d} - (\vec{d} \cdot \vec{a}) \, \vec{a}\right)\\
        B = 2xy &= 2 \, \left(\vec{d} - (\vec{d} \cdot \vec{a}) \, \vec{a}\right) \cdot \left(o - c - ((o - c) \cdot \vec{a}) \, \vec{a}\right)\\
        C = y^2 - r^2 &= \left(o - c - ((o - c) \cdot \vec{a}) \, \vec{a}\right) \cdot \left(o - c - ((o - c) \cdot \vec{a}) \, \vec{a}\right) - r^2.
    \end{align*}

    This gives us the implicit formula for an infinite cylinder.
    In order to \textit{cut} it, we only need to check that the distance between the point on the cylinder and its center isn't greater than:

    \begin{align*}
        \sqrt{radius^2 + \left(\frac{height}{2}\right)^2}
    \end{align*}

    which is the Pythagorean theorem applied for the line going from the center of the cylinder to its end points.

    \section{Deriving the normal vector of the ray-cylinder intersection}

    To compute the normal vector, we need the following elements: $\vecb{i_c}$ the intersection point, $\vec{c}$ the center of the cyclinder and the axis of the cylinder.

    We start by computing the vector $\vec{oi}$ which is the vector between the center of the cylinder and the intersection point:

    \begin{align*}
        \vecb{oi} = \vecb{i_c} - \vecb{c}
    \end{align*}

    Which is then projected onto the axis ($\vecb{oi}'$) in order to find the offset (from the center) of the intersection point along the axis.
    This allows us to compute the vector going from the axis of the cylinder towards the intersection point (on the same "height") which we have to normalize in order to find the normal vector:

    \begin{align*}
        \vecb{normal} = \frac{\vecb{oi} + \vecb{oi}'}{\norm{\vecb{oi} + \vecb{oi}'}}
    \end{align*}

    The last step that is required is to make sure that our normal vector is front-facing, i.e. towards the viewer.
    For this, it must always point in the opposite direction of the view ray direction vector ($\vecb{dir}$).\\

    We recall from our \textit{Linear Algebra} course that "\textit{Two vectors form an acute angle if their dot product is positive}", we apply the same principle in our 3D scenario which means that if $\vecb{dir} \cdot \vecb{norm}$ is positive, we must inverse our normal vector because both are roughly in the same direction.
    Which is simply done by multiplying the normal by the scalar $-1$.

\end{document}

